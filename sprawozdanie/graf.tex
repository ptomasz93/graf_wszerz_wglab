\documentclass[10pt,oneside]{mwbk}

% ustawienia kodowania pliku i języka
\usepackage[T1]{polski}
\usepackage[utf8]{inputenc}

\usepackage{indentfirst}
\usepackage{graphicx}


% czcionka Times
\usepackage{times}

% odstępy na 1.5 (pomimo, iż linespread jest na 1.3
\linespread{1.3}

% dzielenie wyrazów – większe odstępy, mniej dzielenia
\hyphenpenalty=5000
\tolerance=5000

%import z pliku csv
%\usepackage{csvsimple}

%strona tytułowa
\renewcommand {\maketitle}{
\begin {titlepage}
\begin {center}
	\LARGE
	\textbf {PROJEKTOWANIE ALGORYTMOW I METODY SZTUCZNEJ INTELIGENCJI}
	\newline
	\newline
	\textit {SPRAWOZDANIE Z  LABORATORIUM}
	\textbf{ GRAF-przszukiwanie wszerz i w głąb}
	\newline
	\begin{table}
	\begin{center}
	\begin{tabular}{rl}
	IMIĘ I NAZWISKO & Tomasz Piotrowski \\
	NR INDEKSU & 200524 \\	
	TERMIN & czwartek 10:00-12:35 \\
	DATA  & 24.04.2014 \\
	\end{tabular}

	\end{center}
	\end{table}
\end {center}
\end {titlepage}}

\renewcommand*\thesection{\arabic{section}} 
\begin{document}
\maketitle
\section{Wstęp}
	

	\indent  W informatyce grafem nazywamy strukturę G=(V, E) składającą się z węzłów (wierzchołków, oznaczanych przez V) wzajemnie połączonych za pomocą krawędzi (oznaczonych przez E).
Grafy dzielimy na grafy skierowane i nieskierowane:\\
	
	\indent Celem tego ćwiczenia jest zaimplementowanie struktury grafu oraz przetestowanie działania algorytmów przechodzenia w szerz .
	
\section {Graf}

W literaturze można znaleźć dwie propozycję implementacji grafu. Implementacja z pomocą list sąsiedztwa lub za pomocą macierzy sąsiedztwa.
Graf zaimplementowany przezemnie opiera się na dynamicznej tablicy oraz vectorze. Wierzchołki grafu przechowywyane są w tablicy dynamicznej a krawędzie w listach sasiedztwa. Implementacja przezemnie wybrana chrakteryzuje się korzystniejszą złożonością obliczeniową $O(V+E)$.
Macierz sąsiedztwa posiada złożononość obliczeniową $O(V^2)$.
Program przezemnie napisany pozwala na tworzenie grafu za pomocą dostarczonego MENuU, lub odpowiednio wyedytowanych plikow *.txt.
\section {Przeszukiwanie wszerz}
	Przeszukiwanie wszerz (ang. Breadth-first search, w skrócie BFS) – jeden z  algorytmów przeszukiwania grafu. Przechodzenie grafu rozpoczyna się od zadanego wierzchołka s i polega na odwiedzeniu wszystkich osiągalnych z niego wierzchołków. Wynikiem działania algorytmu jest także drzewo przeszukiwania wszerz o korzeniu w s, zawierające wszystkie wierzchołki do których prowadzi droga z s.\\
	\indent Algorytm działa prawidłowo zarówno dla grafów skierowanych jak i nieskierowanych. Algorytm przechodzenie wszerz jest kompletny. Jeśli istnieje droga między zadanymi wierzchołkami, zostanie ona zawsze odnaleziona.\\
	\indent Złożoność czasowa algorytmu związana jest z ilością krawędzi oraz wierzchołków jakie musi przejsc algorytm. W najgorszym przypadku jest to $O(|V|+|E|)$
	\\Zastosowanie algorytmu:
\begin{itemize}
\item odnalezienie wszsytskich połączonych węzłów w grafie.
\item odnalezienie najkrótszej drogi między dwoma wierzchołkami.
\item sprawdzenie czy graf jest dwudzielny
\end{itemize}
\newpage
\section {Przeszukiwanie w głąb}
Przeszukiwanie w głąb (ang. Depth-first search, w skrócie DFS) – jeden z algorytmów przeszukiwania grafu. Przeszukiwanie w głąb polega na badaniu wszystkich krawędzi wychodzących z podanego wierzchołka. Po zbadaniu wszystkich krawędzi wychodzących z danego wierzchołka algorytm powraca do wierzchołka, z którego dany wierzchołek został odwiedzony[1].\\
\indent Złożoność czasowa algorytmu również wynosi $O(|V|+|E|)$ ponieważ tak jak i w przechodzeniu wszerz algorytm odwiedza wszystkie białe wierzchołki \\
\indent Złożoność pamięciowa przeszukiwania w głąb w przypadku drzewa jest o wiele mniejsza niż przeszukiwania wszerz, gdyż algorytm w każdym momencie wymaga zapamiętania tylko ścieżki od korzenia do bieżącego węzła, podczas gdy przeszukiwanie wszerz wymaga zapamiętywania wszystkich węzłów w danej odległości od korzenia, co zwykle rośnie wykładniczo w funkcji długości ścieżki.\\
\indent Algorytm przechodzeni w głąb jest wykorzystywany jako pod program w innych algorytmach działających na grafach.
\begin{itemize}
\item do wyznaczania silnych spójnych składowych grafu skierowanego
\item w algorytmie sortowania topologicznego skierowanego grafu acyklicznego
\end{itemize}
\section{Bibliografia}
\begin{itemize}
\item Thomas H. Cormen, Charles E. Leiserson, Ronald L. Rivest, Clifford Stein: Wprowadzenie do algorytmów.

\end{itemize}

 
\end{document}